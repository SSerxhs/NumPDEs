\documentclass[12pt]{ctexart}
\usepackage{amsfonts,amssymb}
\usepackage{float}
\usepackage{graphicx}
\usepackage{amsmath}
\usepackage{listings}
\usepackage{geometry}
\usepackage[usenames,dvipsnames]{xcolor}
\usepackage{setspace}
\geometry{a4paper}
\geometry{top=2cm} 
\geometry{bottom=2cm}
\geometry{left=0.5cm}
\geometry{right=1cm}

\title{微分方程数值解~项目作业 2}
\author{凌子恒 \\ 信息与计算科学 3200102551}

\begin{document}

\maketitle

\subsection*{原理分析}

利用 Jacobi 迭代法,迭代格式为 $x\gets (1-w)x+wD^{-1}(L+U)x+wD^{-1}b$。

由理论作业结果,取 $w=\dfrac{2}{3}$。

由于矩阵是稀疏矩阵,只需存每行非零列标号和数值,使迭代一次复杂度降至 $O(n)$。

对于 Neumann 边值,由于矩阵奇异,将其中一条式子替换为在 $0$ 处函数值为 $0$。

\subsection*{代码解释}

定义 solution 类存放结果。

solution 类有两个构造函数,分别对应是否给定初值。

构造函数参数分别为二阶导数,边值,段数,迭代次数,精度要求,restriction 模式,interpolation 模式,cycles 模式,估计解(可选),$v_1$(可选),$v_2$(可选)。

其中边值函数返回类型为 unique\_ptr<function\_value> 或 unique\_ptr<derivative\_value>。

实现中,对于 (b),(c),(d) 的选项均以同一类的派生类形式给出,在构造函数中选择对应的模式。V\_cycle 和 FMG 的 $v_1,v_2$ 均可设置,默认为 $2$。

\subsection*{(a)}

代码见 1D.h 和 main.cpp,输出见 1.out,包含了 $e^{\sin x},\sin(\pi x),\dfrac{1}{x+1}$ 的测试。以 $e^{\sin x}$ 为例解释。

以下是 Dirichlet 边值结果

\begin{tabular}{c|c|c|c|c}
 $n$  &$32 $&$64$ &$128$&$ 256$\\
000 &$43 $&$46$ &$48 $&$50$\\
001 &$16 $&$15$ &$15 $&$15$\\
010 &$41 $&$44$ &$46 $&$48$\\
011 &$14 $&$13$ &$13 $&$12$\\
100 &$45 $&$48$ &$50 $&$53$\\
101 &$19 $&$19$ &$18 $&$18$\\
110 &$42 $&$45$ &$47 $&$50$\\
111 &$15 $&$14$ &$14 $&$14$\\
\end{tabular}

其中左侧数字分别为是否为 injection, quadratic, FMG。右侧数字是迭代至 $\epsilon<10^{-8}$ 所需次数。

可以注意到,full\_weighting, quadratic, FMG 效果较好。

接下来的表格为迭代 $15$ 次后的误差表。可以注意到 $15$ 次后已远小于 FD method 的系统误差(error)。

随后是混合边值的结果,不重复叙述。

经过测试,对 $\epsilon>2.2\times 10^{-16}$,均给出了结果。

\subsection*{(b)}

二维情况的代码见 2D.h。这里仅给出线性插值,不支持二次插值。

测速代码见 main2.cpp,输出见 2.out。运行作业 1 代码 $n=64$,运行时间 36299621000,而本代码在 $\epsilon=10^{-8},n=256$ 下仅用时 837094500,$n=64$ 时用时 29686500,速度超过千倍。

\subsection*{(c)}

main.cpp 中将两个类统一作为特化的 IVP<1> 和 IVP<2> 的基类,实现了模板类。

\end{document}
